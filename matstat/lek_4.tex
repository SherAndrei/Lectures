\documentclass[12pt]{article}
 
%Russian-specific packages
%--------------------------------------
\usepackage[T2A]{fontenc}
\usepackage[utf8]{inputenc}
\usepackage[english, russian]{babel}
%for search in russian
\usepackage{cmap}
%--------------------------------------

%Math-specific packages
%--------------------------------------
\usepackage{amsmath}
\usepackage{amssymb}

%Format-specific packages
%--------------------------------------
\usepackage[left=2cm,
            right=2cm,
            top=1cm,
            bottom=2cm,
            bindingoffset=0cm]{geometry}
%--------------------------------------

% for theorems, lemmas and definitions
%--------------------------------------
\usepackage{amsthm}

\newtheorem{definition}{Опр.}
\newtheorem{lemma}{Лемма}

\newtheoremstyle{theorem}    %<name>
                 {\topsep}   %<space above>
                 {\topsep}   %<space below>
                 {\itshape}  %<body font>
                 {}          %<indent amount>
                 {\bfseries} %Theorem head font>
                 {.}         %<punctuation after theorem head>
                 {\newline}  %<space after theorem head> (default .5em)
                 {}          %<Theorem head spec>
\theoremstyle{theorem}
\newtheorem{theorem}{Теорема}

\newtheorem{example}{Пример}

\newtheorem*{remark}{Замечание}
\newtheorem*{corollary}{Следствие}
\newtheorem*{proposition}{Предложение}
%--------------------------------------

% For images
%--------------------------------------
\usepackage{wrapfig}
\usepackage{graphicx}
\graphicspath{ {./} }

%--------------------------------------

% My commands
%--------------------------------------
% for definitions
\newcommand\defin[1]{\textbf{#1}}

\def\R{
    \mathbb{R}
}

\def\E{
    \mathrm{E}
}

\def\D{
    \mathrm{D}
}

\def\P{
    \mathrm{P}
}


%--------------------------------------

\begin{document}
\section{Проверка статистических гипотез}
$X = (X_1, \ldots, X_n)$ имеет плотность вероятности $p(X, \theta)$
по мере $\mu,\ \theta\in\Theta\subseteq\R^1$

\begin{definition}
    Предположение вида $H_0: \ \theta\in\Theta_0$, где
    $\Theta_0\in\Theta$, называется параметрической гипотезой.
    Альтернатива $H_1: \ \theta\in\Theta_1$, где
    $\Theta_1\in\Theta\backslash\Theta_0$
\end{definition}

\begin{definition}
    Если $\Theta_0(\Theta_1)$ состоит из одной точи,
    то гипотеза $H_0$ (альтернатива $H_1$) называется
    простой.
    В противном случае $H_0(H_1)$ - сложная
\end{definition}

\underline{Постановка задачи}:

Необходимо построить правило (статистический критерий),
который позволяет заключить, согласуется ли наблюдение $X$
с $H_0$ или нет.

\underline{Правило.}

Выберем в множестве значений $x$ вектора $X$ (у нас либо
$x = \R^n$, либо $x = N_p \subseteq \R^n$) подмножество
$S$. Если $X \in S$, то $H_0$ \underline{отвергается} и
принимается $H_1$. Если $X \in \overline{S} = x \backslash S$, то
$H_0$ принимается.

\begin{definition}
    Множество $S$ называется критическим множеством или критерием,
    $\overline{S}$ - область принятия гипотезы.
\end{definition}

Возможны ошибки.

\underline{Ошибка 1-го рода} - принять $H_1$, когда
верна $H_0$. Вероятность ошибки 1-го рода $\alpha
 = \P(H_1 | H_0)$ (это условная запись, а не условная вероятность) 

\underline{Ошибка 2-го рода} - принять $H_0$, когда
верна $H_1$. Вероятность ошибки 2-го рода $\beta
= \P(H_0 | H_1)$.

\begin{definition}
    Мощность критерия $S$ называется функция $W(S, \theta) = W(\theta)
    := \P_\theta(X\in S)$ (вероятность отвергнуть $H_0$, когда
    значение параметра есть $\theta$).
\end{definition}

Тогда
\begin{align}
    \alpha &= \alpha(\theta) = W(\theta),\ \theta\in\Theta_0 \\
    \beta  &= \beta(\theta) = 1 - W(\theta),\ \theta\in\Theta_1
\end{align}

Обычно $H_0$ более важна. Поэтому рассматривают критерии
такие, что 
$$\alpha_0 = W(\theta) = \P_{\theta}(X\in S) \leq\alpha \ \forall \theta\in\Theta_0$$
Число $\alpha$ называют уровнем значимости критерия.
Пишут $S_\alpha$ - критерий уровня $\alpha$. Обычно $\alpha$ -
маленькое число, которое мы задаем сами.

\begin{definition}
    Если критерий $S^*_\alpha \in \{S_\alpha\}$ и $\forall\theta\in\Theta_1$ и
    $\forall S_\alpha \ W(S^*_\alpha,\theta) \geq W(S_\alpha, \theta)$,
    то критерий $S^*_\alpha$ называется РНМ-критерием (равномерно наиболее мощным).
\end{definition}

Если $H_0:\theta = \theta_0,\ H_1:\theta = \theta_1$ (то есть
$H_0$ и $H_1$ - простые), то задача отыскания РНМ-критерия
заданного уровня $\alpha$ имеет вид:
$$\P_{\theta_0}(X\in S^*_\alpha) \leq \alpha, \
  \P_{\theta_1}(X\in S^*_\alpha) \geq \P_{\theta_1}(X\in S_\alpha) \ \forall S_\alpha$$

Положим для краткости:
$p_0(x) := p(x, \theta_0),\ \E_0 = \E_{\theta_0},\ p_{1}(x) = p(x, \theta_1),\ \E_1 = \E_{\theta_1}$

Введем множество
$$S(\lambda) = \{x: p_1(x) - \lambda p_0(x) > 0\}, \lambda > 0$$

\begin{theorem}[Лемма Неймана-Пирсона]
    \label{th::lemma_N_P}
    Пусть для некоторого $\lambda > 0$ и критерия $R$
    (когда $X$ попадает в $R$, то $H_0$ отвергается)
    выполнено:
    \begin{enumerate}
        \item \label{th::lemma_N_P::first}  $\P_0(X\in R) \leq \P_0(X\in S(\lambda))$

        Тогда: 
        \item \label{th::lemma_N_P::second} $P_1(X\in R) \leq \P_1(X\in S(\lambda))$
        \item \label{th::lemma_N_P::third}  $P_1(X\in S(\lambda)) \geq \P_0(X\in S(\lambda))$
    \end{enumerate}
\end{theorem}
\begin{remark}
    $X\in S(\lambda) \Leftrightarrow \frac{p_1(x)}{p_0(x)} > \lambda$.
    Так как $p_1(X)$ и $p_0(X)$ - правдоподобие, то критерий
    называется критерием отношения правдоподобия Неймана-Пирсона.
\end{remark}
\begin{remark}
    Утверждение \ref{th::lemma_N_P::third} для $S(\lambda)$
    означает, что
    $$\P(H_1 | H_1) \geq \P(H_1|H_0) \Leftrightarrow W(S(\lambda), \theta_1) \geq W(S(\lambda), \theta_0)$$
    Это свойство назыается несмещенностью критерия $S(\lambda)$
\end{remark}
\begin{proof}
    Дальше для краткости $S(\lambda) = S$. Пусть
    $I_R(x) = \begin{cases}
        1, x\in R \\
        0, x\in 
    \end{cases}$
    \begin{equation}
        \E_0I_R(x) \leq E_0I_S(x)
    \end{equation}

    Докажем пункт \ref{th::lemma_N_P::second}

    Верно неравенство
    \begin{equation*}
        I_R(x)[p_1(x) - \lambda p_0(x)] \leq I_S(x)[p_1(x) - \lambda p_0(x)]
    \end{equation*}

    Действительно, если $(p_1(x) - \lambda p_0(x)) > 0$,
    то $I_S(x) = 1$ и \ref{th::lemma_N_P::second} очевидно.

    Если же $(p_1(x) - \lambda p_0(x)) \leq 0$, то правая часть
    \ref{th::lemma_N_P::second} есть ноль, а левая $\leq$ нуля.

    \underline{Итак, \ref{th::lemma_N_P::second} верно}:

    Интегрируем \ref{th::lemma_N_P::second} по $x\in\R^n$:
    $$\E_1I_R(X) - \lambda\E_0I_R(X) \leq \E_1I_S(X) - \lambda\E_0I_S(X)$$
    \begin{equation}
        \E_1I_S(X) - \E_1I_R(X) \geq \lambda\underbrace{[\E_0I_S(X) - \E_0I_R(X)]}_{\geq 0}
    \end{equation}
    
    Докажем пункт \ref{th::lemma_N_P::third}

    Пусть $\lambda \geq 1$.
    
    Из определения $S\ p_1(x) > p_0(x) \ \forall x\in S.$
    Отсюда
    $$\P_0(X\in S) = \int_{R^n} I_S(X)p_0(x)\mu(dx) \leq \int_{R^n} I_S(X)p_1(x)\mu(dx) = \P_1(X\in S)$$ 
    То есть $\P(H_1 | H_0) \leq \P(H_1 | H_1)$

    Пусть $\lambda < 1$

    Рассмотрим $\overline{S} - \{x: p_1(x) \leq \lambda p_0*x()\}$.
    При $\lambda < 1\ p_1(x) < p_0(x)$ при $x\in \overline{S}$.
    Отсюда
    $$\P_1(X\in \overline{S}) = \int_{R^n} I_{\overline{S}}(X)p_0(x)\mu(dx) \leq \int_{R^n} I_{\overline{S}}(X)p_1(x)\mu(dx) = \P_1(X\in \overline{S})$$ 
    Откуда
    $\P_1(X\in S) \geq \P_0 (X\in S)$
\end{proof}

\begin{example}
    $X = (X_1,\ldots, X_n), \{X_i\}$ - н.о.р., $X_1 \sim N(\theta, \sigma^2)$,
    дисперсия $\sigma^2$ известна. Построим наиболее мощный критерий
    для проверки $H_0: \theta = \theta_0$ против $H_1: \theta = \theta_1$
    (в случае $\theta_1 > \theta_0$). Уровень значимости возьмем $\alpha$.
    \begin{enumerate}
        \item Имеем $p_0 = \left(\frac{1}{\sqrt{2\pi\sigma}}\right)^n \exp{(-\frac{1}{2\sigma^2} \sum^n_{i=1} (x_i -\theta_0)^2)},\ 
        p_1 = \left(\frac{1}{\sqrt{2\pi\sigma}}\right)^n \exp{(-\frac{1}{2\sigma^2} \sum^n_{i=1} (x_i -\theta_1)^2)}$
        $S(\lambda) = \{x:p_1(x) - \lambda p_0(x) > 0\} \Leftrightarrow \exp{}$

        ******

        $(\theta_0 - \theta_1)\sum_{i=1}^n x_i \leq \lambda_2 \Leftrightarrow
        \sum_{i=1}^n x_i > \widetilde{\lambda},\ \widetilde{\lambda}(\lambda) $
        
        *****

        \item Определим $\widetilde{\lambda} = \widetilde{\lambda}_\alpha$ 
            из уравнения
            $$\alpha = \P_{\theta_0}(X \in S(\widetilde{\lambda}_\alpha)) = 
            \P_{\theta_0}(\sum_{i=1}^n X_i > \widetilde{\lambda}_\alpha)$$
            Тогда
            $$\alpha = \P_{\theta_0}\left(\frac{1}{\sqrt{\pi} \sigma} \sum_{i=1}^n(X_i - \theta_0)) > \frac{\widetilde{\lambda}_\alpha - n\theta_0}{\sqrt{\pi}\sigma}\right)=
            1 - \Phi(\frac{\widetilde{\lambda}_\alpha - n\theta_0}{\sqrt{\pi}\sigma})$$
            так как $\frac{1}{\sqrt{\pi}\sigma}\sum (X_i - \theta_0) \sim N(0, 1)$ при $H_0$.
            
            Значит $\Phi(\frac{\widetilde{\lambda}_\alpha - n\theta_0}{\sqrt{\pi}\sigma}) = 1 - \alpha,\
            \Phi(\frac{\widetilde{\lambda}_\alpha - n\theta_0}{\sqrt{\pi}\sigma}) = \xi_{1-\alpha}$
            $\xi_{1 - \alpha}$ - квантиль станд. норм. закона уровня $1 - \alpha$.
            Окончательно, $\widetilde{\lambda}_\alpha = n\theta_0 + \sqrt{\pi}\sigma \xi_{1-\alpha}$
        \item Положим $S^*_{\alpha} = \{x: \sum_{i=1}^n x_i > \widetilde{\lambda}_\alpha\}$
        
        *******

        Так как $S^*_{\alpha}$ не зависит от $\theta_1$,
        то $S^*_{\alpha}$ - РНМ-критерий для $H_0: \theta = \theta_0$
        против $H^+_1 : \theta > \theta_1$
        Мощность критерия $S^*_{\alpha}$ для $H_0$ при альт. $H^+_1$

        $$W(\theta, S^*_{\alpha}) = \P_\theta\left(\sum_iX_i > n\theta_0 + \sqrt{n}\sigma\xi_{1-\alpha}\right) = $$
        $$ = \P_\theta\left(\frac{1}{\sqrt{n}\sigma} \sum_i(X_i - \theta) > \frac{\sqrt{n}(\theta_0 - \theta)}{\sigma} + \xi_{1-\alpha}\right) =
        1 - \Phi\left(\xi_{1-\alpha} + \frac{\sqrt{n}(\theta - \theta_0)}{\sigma}\right)$$
    \end{enumerate} 
\end{example}


\end{document}