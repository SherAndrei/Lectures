\documentclass[12pt]{article}
 
%Russian-specific packages
%--------------------------------------
\usepackage[T2A]{fontenc}
\usepackage[utf8]{inputenc}
\usepackage[english, russian]{babel}
%for search in russian
\usepackage{cmap}
%--------------------------------------

%Math-specific packages
%--------------------------------------
\usepackage{amsmath}
\usepackage{amssymb}

%Format-specific packages
%--------------------------------------
\usepackage[left=2cm,
            right=2cm,
            top=2cm,
            bottom=2cm,
            bindingoffset=0cm]{geometry}
%--------------------------------------

% for theorems, lemmas and definitions
%--------------------------------------
\usepackage{amsthm}

\counterwithin*{equation}{section}

\newtheorem{definition}{Опр.}
\newtheorem{lemma}{Лемма}
\newtheorem*{remark}{Замечание}
\newtheorem*{corollary}{Следствие}
\newtheorem*{proposition}{Предложение}
\newtheorem*{example}{Пример}

\newtheoremstyle{basic_theorem}    %<name>
                 {\topsep}   %<space above>
                 {\topsep}   %<space below>
                 {\itshape}  %<body font>
                 {}          %<indent amount>
                 {\bfseries} %Theorem head font>
                 {.}         %<punctuation after theorem head>
                 {\newline}  %<space after theorem head> (default .5em)
                 {}          %<Theorem head spec>
\theoremstyle{basic_theorem}
\newtheorem{theorem}{Теорема}

\newtheoremstyle{name_theorem}
                {\topsep}
                {\topsep}
                {\itshape}
                {}
                {\bfseries}
                {}
                {\newline}
                {\thmnote{#3}}
\theoremstyle{name_theorem}
\newtheorem*{named_theorem}{Теорема}
%--------------------------------------

% For images
%--------------------------------------
\usepackage{wrapfig}
\usepackage{graphicx}
\graphicspath{ {./images/} }

%--------------------------------------

% My commands
%--------------------------------------
% for definitions
\newcommand\defin[1]{\textbf{#1}}

\def\eps{
    \varepsilon
}
\def\Eps{
    \mathcal{E}
}

\def\R{
    \mathbb{R}
}

\def\E{
    \mathrm{E}
}

\def\D{
    \mathrm{D}
}

\def\P{
    \mathrm{P}
}

\def\littleO{
    \overline{\overline{o}}
}

%--------------------------------------

\begin{document}
    Схема засорений Мартина-Йохаи имеет вид:
    $$y_t = u_t + z^\gamma_t\xi_t,\ t = 1, \ldots, n$$

    Здесь $\{u_t\}$ - "полезный сигнал" (временной ряд),
    $\{z_t^\gamma\}$ - н.о.р. сл.в., $z_1^{\gamma^-} \sim Bin(1, \gamma)$
    с $0 \leq \gamma\leq1$ ($\gamma$ - уровень засорения);

    $\{\xi_t\}$ - н.о.р. сл.в. - грубые выбросы, $\xi_1$ - имеет распределение
    $\mu_\xi\in M_\xi$; распределение $\mu_\xi$ неизвестно, а множество $M_\xi$
    известно;

    Последовательность $\{u_t\}, \{z^\gamma_t\}, \{\xi_t\}$ независимы между собой.

    Пусть $y_1, \ldots, y_n$ - наблюдения, и распределение
    вектора $Y_n=y_1, \ldots, y_n$ висит от неизвестного параметра $\beta$.
    Пусть $\hat{\beta}_n$ - некоторая оценка $\beta$

    \underline{Основное предположение}

    При любом $0 \leq \gamma\leq1$ существует предел
    $$\hat{\beta}_n\xrightarrow{\P}\theta_\gamma,\ n\rightarrow\infty;\theta_0=\beta$$

    \begin{definition}
        Если существует предел
        $$IF(\theta, \mu_\xi):=\lim_{y\rightarrow}$$
    \end{definition}


    ******

    Если функционал влияния существует, то 
    $$\theta_\gamma = \theta_0 + IF(\theta_\gamma, \mu_\xi)\gamma + o(\gamma),\ \gamma\rightarrow+0$$
    То есть $IF(\theta_\gamma, \mu_\xi)$ характеризует главный линейный по
    $\gamma$ член в разложении по $\gamma$ асимптотического смещения $\theta_\gamma - \theta_0=\theta_\gamma-\beta$

    \begin{definition}
        Величина $GES(\theta_\gamma,M_)$
    \end{definition}

    ******

    \begin{example}
        $$\begin{cases}
            u_t &= a + \xi_t \\
            y_t &= u_t + z^\gamma_t\xi_t, \ t=1,\ldots,n
        \end{cases}$$
    \end{example}
    $\{\xi_t\}$ - н.о.р. сл.в., $\E\eps_1=0$ (тогда $\E u_t=a$),
    $\E\vert\xi_1\vert<\infty$
    
    Тогда $\overline{y}\xrightarrow{\P}E(u_1+z_1^\gamma\xi_1) = a + \gamma\E\xi_1=\theta^{LS}_\gamma$

    Функция $\theta_\gamma^{LS}$ определена при всех $\gamma$,
    $$\frac{d\theta_\gamma^{LS}}{d\gamma}=\E\xi_1=IF(\theta_\gamma^{LS}, \mu_\xi)$$.
    Если $M_1$ - ласс распределений с конечным первым моментом, то
    $$GES(\theta_\gamma^{LS}, M_1) = \sup_{\mu_1\in M_1} \vert\E\xi_1\vert = \infty$$
    Оценка $\overline{y}$ е $B$ - робастна на  классе $M_1$

    \begin{example}[Выбороная медиана]
        Пусть $u_t = a+\eps_t,\ t=1,\ldots,n$, где
        $\{\eps_t\}$ - н.о.р. сл.в.,
    \end{example}
    *******

    Мы знаем, что если $G(x)$ дифф. в нуле, и $g(0) = G'(0)>0$,
    то для выборочной медианы справедлива асимп. нормальность:
    $$n^{1/2}(\hat{m}_n - a)\xrightarrow{d}N(0, \frac{1}{n}g^2(0)), n\rightarrow\infty$$
    
    *****************

    Пусть $\u_t$

\end{document}