\documentclass[10pt]{article}
 
%Russian-specific packages
%--------------------------------------
\usepackage[T2A]{fontenc}
\usepackage[utf8]{inputenc}
\usepackage[english, russian]{babel}
%for search in russian
\usepackage{cmap}
%--------------------------------------

%Math-specific packages
%--------------------------------------
\usepackage{amsmath}
\usepackage{amssymb}

%Format-specific packages
%--------------------------------------
\usepackage[left=2cm,
            right=2cm,
            top=1cm,
            bottom=2cm,
            bindingoffset=0cm]{geometry}
%--------------------------------------

% for theorems, lemmas and definitions
%--------------------------------------
\usepackage{amsthm}

\newtheorem{definition}{Опр.}
\newtheorem{example}{Пример}[definition]
\newtheorem{lemma}{Лемма}

\newtheoremstyle{theorem}    %<name>
                 {\topsep}   %<space above>
                 {\topsep}   %<space below>
                 {\itshape}  %<body font>
                 {}          %<indent amount>
                 {\bfseries} %Theorem head font>
                 {.}         %<punctuation after theorem head>
                 {\newline}  %<space after theorem head> (default .5em)
                 {}          %<Theorem head spec>
\theoremstyle{theorem}
\newtheorem{theorem}{Теорема}

\newtheorem*{remark}{Замечание}
\newtheorem*{corollary}{Следствие}
\newtheorem*{proposition}{Предложение}
%--------------------------------------

% My commands
%--------------------------------------
% for definitions
\newcommand\defin[1]{\textbf{#1}}

\def\R{
    \mathbb{R}
}

\def\E{
    \mathrm{E}
}

\def\D{
    \mathrm{D}
}

\def\P{
    \mathrm{P}
}

\def\F{
    \mathcal{F}
}

\def\B{
    \mathfrak{B}
}

%--------------------------------------

\begin{document}
    \section{Вступление}
    Собственно - акции. У банка различные вклады
    интвестирование капитала, вложения в ценные бумаги, учреждения дочерних фирм (не коммерческие компании)
    Прибыль относительно маленькая у страховых компаний, она заложена в тарифную ставку,
    дивиденты по акциям. У банков аналогично + ставка по кредитам
    Цель клиента банка - накопление
    Два типа страховых компаний
    \begin{enumerate}
        \item Лайф еншуранс - страхование жизни
            Долгосрочные типы страхования
            Типы: пенсионные, жизни, 
            предполгается либо единая выплата, либо рента.
        \item Нон-лайф иншуранс - страхование не жизни
            Страхование имущества, грузов, средств транспорта (самолеты, корабли),
            страхование технических рисков, механизмы
            Медицинское страхование, страхование финансов (невозвращение кредитов)
            Страхование финансовых гарантий и страхование акций, страхование гражданской и 
            профессиональной ответственности (допустим, нотариуса), наконец страхование автотранспорта и тд
        \end{enumerate}
    Обозначения:
    Страховая компания - страховщик, клиент страховой компании - страхователь.
    Заключается договор - оформляется страховой полис.
    Страховой случай - то, что 
    Страховая премия - то, что платит страхователь по поводу заключаемого контракта
    Страховая сумма - максимальная сумма, которая должна быть выплачена в описанном случае.
    Страховое возмещение - неполная сумма, по выше сказанному.

    Первый страховой полис был заключен между Генной и Майоркой.
    СТраховой взнос в случае потери должен быть воззвращен в двойном размере
    Первая страховая компания была создана 1914 году
    Страховой отдел в России отдел в 1858 году в Москве и далее во многих других городах России.

    В Росси начинает быстро и успешно развиваться актуарная деятельность
    Выходят многие книги
    Сделал копию, в библиотеке на 14 этаже стоят 
    В этих книгах 4 включены очень серьезные новые результаты
    обратившие на себя внимание западных стран

    Книга Савича в 2003 году издали еще раз эту книгу для страховых компаний, так чтобы они 
    знали что было  в России в начале ХХ века.
    Актуарное дело развивалось настолько серьезно, что в 14 году был намечен
    актуарный конгресс. Первая мировая помешала этому конгрессу
    До революции актуарное дело развивалось очень и очень серьезно.
    Итак после революции в России осталась одна огромная компания - Россгострах (ранее просто Госстрах).
    Была еще одна - Ингоссстрах - которая занималась на внешнем рынке.
    Актуарная деятельность была и развивалась.
    Нельзя сравнивать с тем, что было до 17 года.
    К началу 90х годов в воздухе чувствоввалось, что надо начинать деятельность по поводу
    подготовки актуариев.
    На мехмате открылось направление по подготовке математиков к актуарной математике
    Лекции имели колоссальный интерес. Люди стояли у стен. То, что рассказывалось в лекциях, нигде найти было
    невозможно.
    В последние годы на трудовом рынке большие сложности
    

        Страхование жизни предполгает жизнь без ситуаций, связанных с резким вымиранием большого количества людей.
        Наибольшее влияние оказывает возраст.
        Страховать шахтера или преподавателя на случай смерти, то, конечно, профессия первого куда
        более опасна, чем второго. Соответствующие таблицы будут создаваться отдельно
        для разных категорий.

    Клиент $\xrightarrow[\mbox{на случ. появл. комп., маловероятн., но риск. ситуац.}]{\mbox{финансовая гарантия}}$ Страхование

    \begin{definition}
        \defin{Вероятность выживания}
        Рассмотрим однородную группу людей,
        Обозначим $x$  - возраст нашего клиента. 
        обозначим через $T_x$ остаточное время жизни - случайная величина.

        $\omega$ - максимальный возраст жизни ($\approx$ 100 лет)

        Функция распределения - $F(t) = \P(T_x \leq t) := {}_tq_x$

        ${}_tp_x = \P(T_x > t)$ - с момента $x$ человек проживет более, чем $t$ лет.

    \end{definition}

    \begin{definition}
        Вероятность смерти (Вероятность, что на промежутке $(t, t + t')$ клиент скончается): 
        $${}_{\frac{t}{t_x}}q_x = \P(t < T_x \leq t + t')$$

    \end{definition}
    Вероятность, что человек проживет больше, чем $t$: 
    $$\P(T_x > t) = \P(t < T_x \leq t + t') + \P(T_x > t + t')  \Rightarrow {}_tp_x = {}_{\frac{t}{t_x}}q_x + {}_{t+ t'}p_x$$

    \begin{definition}
        Интенсивность смерти в момент $t$, при условии, что $T_x > t$:
        $$\P(t < T_x \leq t + \Delta | T_x > t) = \frac{\P(t < T_x \leq t + \Delta t)}{\P(T_x > t)} = \frac{F(t + \Delta t) - F(t)}{{}_tp_x}$$
        $$\lim_{\Delta t \to \infty} \frac{\P(t < T_x \leq t + \Delta | T_x > t)}{\Delta t} = \frac{F'(t)}{{}_tp_x} = \frac{f(x)}{1-F(t)} := \mu_x(t) $$
    \end{definition}
        $$\mu_x(t) = \frac{F'(t)}{1-F(t)} = \frac{\frac{\partial}{\partial t}({}_tp_x)}{{}_tp_x} = \frac{d}{dt} ln({}_tp_x)$$
        $$\boxed{{}_tp_x = e^{-\int_0^t \mu_x(u) du} }$$

    Модели смертности
    \begin{enumerate}
        \item Муавр
        Средняя продолжительность жизни около 86 лет.
        \item Gomretz
        Предполгает, что человек может жить до бесконечности.
        \item Makeham
        $$\mu_x(t) = D + AC^{x+t}$$
        $${}_t{}p_x = e^{-Dt - \frac{AC^x}{lnC}(C^t - 1)}$$
    \end{enumerate}
    Пусть $L_x$ - число людей возраста $x$.
    $\chi_i(t) = \begin{cases}
        1 \quad > t     \\
        0 \quad \leq t  \\
    \end{cases} $
\end{document}